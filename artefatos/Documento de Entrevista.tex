\documentclass{article}
\usepackage[utf8]{inputenc}

\title{Documento de Entrevista}

\begin{document}
\maketitle

\section{Informações da Reunião}

\begin{itemize}

	\item Grupo de Desenvolvimento: Grupo 1 (Disciplina)
	\item Dia: 18/07/2017 (segunda-feira)
    \item Horário: 10h-12h
    \item Cliente Consultado: André Brito

\end{itemize}

\section{Participantes}

\begin{itemize}

	\item Lucas de Carvalho Gomes: Desenvolvedor
    \item Renan Basílio: Desenvolvedor
    \item Vinícius Almeida Alves: Desenvolvedor
    \item André Brito: \textit{Stakeholder}

\end{itemize}

\section{Notas da Reunião}

A reunião teve o intuito de discutir os requisitos e regras de negócio do sistema, uma vez que a equipe já havia elaborado uma versão inicial do Documento de Requisitos. Abaixo, estão as informações obtidas com o \textit{stakeholder}:

\begin{itemize}

	\item Formato do código identificador de uma disciplina: formado por três letras seguidas de três números.
    
    \item Carga horária e créditos de uma disciplina: disciplinas devem ter um campo representando a carga horária no banco de dados. A partir desta, determina-se a quantidade de créditos fornecida pela disciplina.
    
    \item Notas de disciplinas: serão dadas em conceitos (letras), indo de A a D, sendo A o menor conceito e D o menor. Um conceito D implica em uma reprovação na disciplina.
    
    \item Existência de disciplinas obrigatórias: embora existam cursos que exijam isso e ainda que a reprovação implique em obrigatoriedade, não será necessário implementar a obrigatoriedade de cursar uma disciplina no sistema.
    
    \item Histórico e Boletim: nenhuma operação que envolva esses documentos deve ser realizada no sistema.
    
    \item Pré-requisitos de disciplinas: pode-se assumir que nenhuma disciplina depende de outras disciplinas.
    
    \item Disciplinas obrigatórias fora da área de pesquisa não precisam ser implementadas.
    
    \item Cadastro de cursos (especializações): não será necessário fazer. Pode-se assumir que os cursos já existem.
    
    \item Operações de CRUD (Create, Read, Update and Delete) em um banco de dados devem ser implementadas para a grade curricular de um curso.
    
    \item Alunos que já foram aprovados em uma disciplina não podem cursá-la novamente.

\end{itemize}



\section{Ações Futuras}

Devem ser criadas regras de negócio no Documento de Requisitos que estabeleçam o formato do código de uma disciplina, a relação entre quantidade de créditos de uma disciplina e a sua carga horária, as possíveis notas de uma disciplina, a reprovação no caso de a nota ser "D" e a impossibilidade de cursar uma disciplina na qual já existe uma aprovação. 

Também deve-se criar um requisito que capture as operações de CRUD para a grade curricular de um curso.

\end{document}
